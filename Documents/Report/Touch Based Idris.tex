\documentclass[oribibl]{llncs}
\pagestyle{headings} %page numbers
\usepackage{makeidx}  % allows for indexgeneration
\usepackage[utf8]{inputenc}
\usepackage[table,xcdraw]{xcolor}
\usepackage{multirow}
\usepackage{pdfpages}
\usepackage{float}
\usepackage{enumitem}
\usepackage[T1]{fontenc} % Fixed missing font warning for \maketitle
\usepackage{url}
\usepackage{todonotes} % remove later - this is for TODO notes
%\usepackage{amsmath,amssymb}  We don't use this?
\usepackage{url}
\usepackage{alltt}

% \hypersetup{%
%   citecolor=black
% }

\bibliographystyle{plain}

\begin{document}
\mainmatter{}
\title{Touch Based, Visual Syntax for Idris}
\author{Nicolai Dahl Blicher-Petersen \and Christian Harrington \\
\email{\{ndbl, cnha\}@itu.dk}}
\institute{IT University of Copenhagen, Rued Langgaards Vej 7, 2300 Copenhagen S, Denmark}

\maketitle

\begin{abstract}
In dependently typed programming languages, the type system can help provide powerful tool support, allowing for a faster and more
convenient way of programming e.g. by means of automatic case splitting and code-inference. Touch-based programming interfaces are often
cumbersome to use due to the heavy reliance on the virtual keyboard, so
minimizing its use could potentially benefit the programmer tremendously.

In this report we investigate the potential of the dependently typed
programming language, Idris, in the field of touch-based programming.
We propose a more visual, high-level syntax for Idris and describe our prototype of a structured editor
that supports this syntax and runs on the Apple iPad.

Through a thorough analysis of existing solutions, usability tests, and
heuristics evaluations, we evaluate our prototype and present a road map for
future work on the editor.

\todo{Insert what we conclude}

\keywords{VPL, Idris, Structured Editor, Usability}
\end{abstract}
%!TEX root = ../Touch Based Idris.tex
\chapter{Introduction}
\label{sec:Introduction}
In the last few years touch-based interfaces, in the form of smart phones and tablets, have proliferated. These devices often rely on virtual keyboards for text-input, which can be cumbersome to use.
This is especially an issue when programming, as this is a task traditionally performed with a physical keyboard and mouse.
To our best knowledge, no widespread solution for programming on a touch-based interface exists.

Idris is ``a general purpose functional language with dependent types''\,\cite{brady2013idris}. The expressive nature of the Idris type system makes advanced tool-support possible, such as automatically generating pattern matching on the different possible constructors for a term, or even filling out a term automatically.

The goal of this project is to develop a design for a touch-based programming editor, that leverages Idris to improve the programming experience compared to existing solutions, e.g.\ by minimizing use of the virtual keyboard.
The design will target the Apple iPad, as it is the most prolific tablet computer available today.

To accomplish this, we will first study existing solutions for touch-based programming, together with other programming paradigms that might improve usability for a touch-based design.
We aim to identify and define their usability shortcomings by using existing usability theories.

With the knowledge we have gained from these activities, we will present a set of goals and requirements for our design.
Based on these, we will present a series of design iterations, with changes rooted in usability tests and usability theory.
To further our understanding of the usability implications of our design, a prototype application will be developed.
The last design we present will serve as a base for future development.

\section{Overview}
\todo{Write overview}

%!TEX root = ../Touch Based Idris.tex
\section{Background}
\label{sec:Background}
\subsection{Dependent Types}
In many programming languages, such as Java or Haskell, a type can depend on another type, such as \texttt{ArrayList<A>} in Java, or \texttt{[a]} in Haskell. By suppling a concrete type, we can say that a collection only holds this specific type of values, e.g. \texttt{ArrayList<String>} or \texttt{[String]}. But in a programming language with dependent types, it is possible to go further. With dependent types, a type can be dependent on \emph{values} as well as types. This allows for increased expressiveness in the type system, which can be used for ensuring correctness, conducting proofs, and improving tool support. For this project, it is the last point we will focus on. Dependent types have been implemented in several programming languages, such as Coq\todo{Ref}, Agda\todo{Ref}, and Idris, which we will examine next.

\subsection{Idris}
\label{subsec:Idris}
Idris is a dependently typed programming language, initially developed by Edwin Brady~\cite{Idris}. The syntax is heavily inspired by Haskell. Let us start by looking at a simple example. In Figure~\ref{fig:nat}, a data type representing the natural numbers is defined. It consists of two constructs \texttt{Z}, representing zero, and \texttt{S~n}, representing the successor of the natural number \texttt{n}.

\begin{figure}
\begin{alltt}
data Nat : Type where
  Z : Nat
  S : Nat \(\to\) Nat
\end{alltt}
\caption{The natural number data type.}
\label{fig:nat}
\end{figure}

The definition for \texttt{Nat} looks similar to how it would look in Haskell. Let us move to a slightly more complex example, where we make use of dependent types. In Figure~\ref{fig:vect}, the dependent vector, \texttt{Vect: Nat $\to$ Type $\to$ Type} is defined. This type takes two arguments: a \texttt{Nat}, \texttt{n}, describing the length of the vector, and a \texttt{Type}, \texttt{a}, specifying the type of the elements it contains.

\begin{figure}
\begin{alltt}
data Vect : Nat \(\to\) Type \(\to\) Type where
  Nil  : Vect Z a
  (::) : a \(\to\) Vect k a \(\to\) Vect (S k) a
\end{alltt}
\caption{The vector data type.}
\label{fig:vect}
\end{figure}

The first construct, \texttt{Nil}, tells us that the empty vector always has a length of \texttt{Z}, or zero. The second constructor, \texttt{(::)} or cons, takes a new element of type \texttt{a}, and prepends it to a vector holding elements of type \texttt{a}. The length of the resulting vector is increased by one compared to the previous vector. We say the vector is parameterized by the type \texttt{a}, as it is the same in both constructors. However, the length of the list, specified by the natural number \texttt{n}, is different in each constructor, so we say the vectors is a family of data types \texttt{indexed} by the natural numbers.

But why record the length of the vector in its type? How does this help us? A simple example of how this can be useful can be seen in the \texttt{zip} function, shown in Figure~\ref{fig:zip}.

\begin{figure}
\begin{alltt}
zip : Vect n a \(\to\) Vect n b \(\to\) Vect n (a, b)
zip Nil       Nil       = Nil
zip (x :: xs) (y :: ys) = (x, y) :: zip xs ys
\end{alltt}
\caption{Zip function for vectors in Idris.}
\label{fig:zip}
\end{figure}

The \texttt{zip} function takes two arguments. Each argument is a vector of the same length \texttt{n}. \texttt{n}, \texttt{a}, and \texttt{b} are inferred implicitly from their context. These arguments could also be explicitly declared as implicit by surrounding them with curly braces: \texttt{\{n:~Nat\}~$\to$ \{a:~Type\}~$\to$ \{b:~Type\}~$\to$ Vect~n~a~$\to$ Vect~n~b~$\to$ Vect~n~(a,b)} The function zips these two vectors together, producing a vector of the same length, containing tuples \texttt{(a, b)} of the values from the input vectors. If one tries to compile a program that attempts to zip two vectors of different lengths, the type checker will report it as an error, as opposed to failing at runtime. Notice how it is unnecessary to match on the cases where one vector has elements while the other is empty. This situation is impossible, due to the types. In fact, matching on this case is ill-typed, as \texttt{Z} and \texttt{S~n} cannot unify.\todo{Should we mention dependent pairs and the with-rule, when we don't support them?} For our final example, let us look at the \texttt{filter} function in Figure\ref{fig:filter}.

\begin{figure}
\begin{alltt}
filter : (a \(\to\) Bool) \(\to\) Vect n a \(\to\) (p ** Vect p a)
filter p [] = (_ ** [])
filter p (x::xs) with (filter p xs)
  | (_ ** tail) =
    if p x then (_ ** x::tail) else (_ ** tail)
\end{alltt}
\caption{Filter function for vectors in Idris.}
\label{fig:filter}
\end{figure}

The \texttt{filter} function uses two constructs that we have not seen so far, dependent pairs and the with rule. Dependent pairs let us specify a type that is dependent on another value. In \texttt{filter}, we do not know beforehand what length the resulting vector will have, so instead we return a dependent pair, where the first element is the length, and the second element is a \texttt{Vect} of that length. Not knowing the length is also a problem in the cons case of \texttt{filter}. If the head of the \texttt{Vect} passes the predicate \texttt{p}, we want to cons the tail of the \texttt{Vect} onto the head. But since we do not know what the length of \texttt{filter p xs} is beforehand (as it depends on how many elements passes the predicate \texttt{p}), we cannot construct the type for the resulting vector. But by using the \texttt{with} keyword, we can pattern match on the result of \texttt{filter p xs}. We know the length of \texttt{tail}, since we have filtered it already, so we can construct the resulting \texttt{Vect} without any problems. The underscores (eg. \texttt{(\_ ** tail)}) tell Idris to infer the value.
%!TEX root = ../Touch Based Idris.tex
\chapter{Analysis of Existing Solutions}
\label{sec:Analysis}
In this section, we analyze a range of touch-based programming editors, some structured editors, as well as a few visual programming languages (VPLs) and their editors.
The goal of this analysis is to gain a better understanding of what works and what does not work in existing solutions.
This knowledge will help us in setting realistic and attainable goals and requirements for our own solution, which will be discussed in Section~\ref{sec:GoalsAndRequirements}.

\subsection{Investigation Scope}
There is already a large of amount of visual and touch-based programming editors, as can be seen from a list compiled by Eric Hosick\,\cite{hosick2014}.
For this reason it is important for us to define which types of existing solutions that are of interest.

As Idris is a general purpose programming language, we will not be investigating domain specific languages or platforms, unless elements of their syntax or integrated development environment (IDE) has potential of working with general purpose languages.

While there are many different VPLs, they often fall into certain categories, such as the ``boxes and wires'' visual workflow languages, such as G, used in Labview.
Where it is appropriate, we will choose a single representative from a category, as we are more interested in getting a broad view than going in depth.
As we are designing a touch-based editor, our focus will be on other touch-based editors. Visual editors will not be examined with the same detail.

With these limiting factors there are still too many candidate solutions for us to analyze and evaluate them all. 
The following analysis consists of a selection of the existing solutions that we found most relevant to our study and mainly applies the heuristic evaluation technique by Nielsen\,\cite{nielsen1990heuristic}. 
Such an evaluation can give an idea of possible usability issues but it will not help discover them all. 
This suits our purpose, as it is our mission to get an overview of the major pitfalls as well as what generally works well.

The thoroughness of our analyses also depends on something as practical as whether we have been able to install the tool/language on our devices.
Some of the solutions in question are old or experimental.
In some cases the only available material is conceptual descriptions by the creators.

\subsection{Touch-based Editors}
\label{subsec:TouchBasedEditors}
We will start by looking at a series of touch-based editors that try to emulate more traditional editors, that are used with a keyboard and mouse. 

\subsubsection{CodeToGo}
\label{subsub:CodeToGo}
CodeToGo claims to be the first app for iOS in which you can write and run code in your favorite language\,\cite{codetogo}. It is backed by the \url{ideone.com} website\cite{ideone} that remotely evaluates the code, when the user presses ``Run'' (See Figure~\ref{fig:CodeToGo_screenshot}). The app provides shortcuts for the most commonly used characters and even lets you customize which ones to have easiest access to in which language. The usability considerations stop here though. CodeToGo does not take advantage of the touch based interface but tries to overcome it. While there is syntax highlighting there is no code completion or static checking, which quickly makes programming a cumbersome task.

\begin{figure}
	\centering
		\includegraphics[width=80mm]{diagrams/CodeToGo_screenshot.PNG}
	\caption{The upper right corner of an example Scala program in CodeToGo.}
\label{fig:CodeToGo_screenshot}
\end{figure}

The most severe usability problem for CodeToGo is the low degree to which the user is able to recover from errors\,\cite{nielsen1990heuristic}. 
If you have a syntax error in your code you will get a standard console compile error from the \url{ideone.com} server. 
This error contains the line number where your program failed, but when you dismiss the error you will have to remember this line number and manually count your way down to the line where the mistake was, as the editor does not display line numbers\,\cite{nielsen1990heuristic}.

The editor only has the aforementioned character selector as an accelerator for advanced users. Other than that there are no ways for users to improve when using the tool other than to learn to type faster on an iPad\,\cite{nielsen1990heuristic}.

Nielsen recommends that you follow the iOS platform standards so that the user does not have to put too much effort into learning how to use each app in a special way. Nielsen also recommends having undo support. CodeToGo does have undo support and follows the standard way of iOS, which is to shake the device. One could argue that shaking their iPad is not the most elegant way of allowing programmers to undo their typing, so in this case following the standards is not necessarily the best way to go for a mobile programming interface.

\paragraph{Takeaways}
\begin{itemize}
	\item \textbf{Ta-1}: Do not assume that a virtual keyboard is as usable as a physical one\,\cite{nielsen2013mobile}. The touch interface has potential if you design your user experience to take advantage of it, but if you chose to ignore its potential/limitations you will have lower usability\,\cite{nielsen1990heuristic}.
	\item \textbf{Ta-2}: You need accelerators for users to become faster and more comfortable with the interface\,\cite{nielsen1990heuristic}. An accelerator could, for example, be auto completion and/or static checking.
	\item \textbf{Ta-3}: Undo support is essential for usability\,\cite{nielsen1990heuristic}, but shaking the device is probably not the right input method.
\end{itemize}

\subsubsection{Textastic}
\label{subsub:Textastic}
Like CodeToGo, Textastic aims to be a general purpose programming editor, and as such supports many popular programming languages. Unlike CodeToGo, however, it does not support any way to run your code (besides manually copying your program to a computer and running the code there). Textastic is interesting due to an interface component that is not seen in CodeToGo.

\begin{figure}
	\centering
		\includegraphics[width=110mm]{diagrams/textastic_keyboard_screenshot.png}
	\caption{The Textastic keyboard with special characters easily accessible.}
\label{fig:textastic_keyboard_screenshot}
\end{figure}

Textastic uses a shortcut bar that can be seen in Figure~\ref{fig:textastic_keyboard_screenshot}. This bar allows quick access to commonly used characters, that would otherwise be hard to get to with the virtual keyboard. To select a special character you tap and hold your finger on the specific button and swipe towards the character that is needed. 

The biggest problem with Textastic is that you have no way of knowing whether your code will run or not\,\cite{nielsen1990heuristic}.\ On a computer, this would not be as big of a problem, in fact many programming editors do not analyze a program's semantics, instead they rely on other software to run the code. This is much harder on a device such as an iPad, where multitasking is not as prevalent. The text editing interface is very reminiscent of none-touch GUI text editors.

\paragraph{Takeaways}
\begin{itemize}
	\item \textbf{Ta-4}: The solution Textastic has come up with for quickly accessing a large collection of symbols is the most intuitive we have seen.
	\item \textbf{Ta-5}: It is not enough to have a good editing interface, if such a basic issue as running your code is unaddressed.
\end{itemize}


\subsubsection{Raskell}
\label{subsub:Raskell}
Raskell is a Haskell editor for the iPad and a part of the UI can be see in Figure~\ref{fig:Raskell_screenshot}. It is based on the Hugs implementation of Haskell 98, and includes the ability to interpret your programs locally. The text editor itself is inspired by Vi, featuring many of the same keyboard commands. When you have written your program, you can load it into the interpreter, which gives you a read–eval–print loop (REPL), letting you test out your program. Raskell supports syntax highlighting for Haskell, and features most libraries for Hugs, but does not have auto completion.

The text editing itself works well, but a lack of auto completion and snippets mean you will be using the virtual keyboard to a large extent, which is generally a problem according to Nielsen\,\cite[pp. 76]{nielsen2013mobile}. 

\begin{figure}
	\centering
		\includegraphics[width=110mm]{diagrams/Raskell_screenshot.png}
	\caption{The top of the Raskell environment with code editing to the right and a REPL to the
	left.}
\label{fig:Raskell_screenshot}
\end{figure}

Apart from this issue, the edit-compile-test loop is flexible and efficient. Especially compiler errors are handled in an elegant way, letting the user recover from syntactical errors\,\cite{nielsen1990heuristic}. If a compile error occurs, the user is presented with a split-screen view showing the error and a line number. The REPL is also extremely useful for trying out parts of your program.

Compared to CodeToGo, the Raskell development flow is quicker. This may be due to the local interpreter that evaluates your Haskell code as opposed to the network based approach CodeToGo takes.

\paragraph{Takeaways}
\begin{itemize}
	\item \textbf{Ta-6}: A good edit-compile-test loop greatly improves the development flow.
	\item \textbf{Ta-7}: A solid way to recover from syntax errors is essential to a good development experience\,\cite{nielsen1990heuristic}.
\end{itemize}

\paragraph{}

All three editors in this section have one main problem: The heavy reliance on the virtual keyboard. Also, only using one type of touch-gesture (the single tap) seems like a wasted opportunity.
According to Nielsen, it is important to optimize your solution for the touch
interface and not simply convert an existing desktop interface\,\cite[p 26, p
41]{nielsen2013mobile}.

One last issue that all of the above editors have in common is the lack of auto completion and user accelerators in general.

\subsection{Structural, Touch-Based Editors}
In this section we look at structured editors, which are different from the previously mentioned editors in that the editors are cognizant of the underlying structure of the program, the abstract syntax tree (AST). The user more or less directly manipulates this AST\@.

\subsubsection{Lisping}
\label{subsub:Lisping}
Using the Scheme dialect of LISP you can program and execute code right on your iPad with Lisping. The idea is to take advantage of the close proximity between the abstract syntax tree and source syntax of LISP to edit code in a structural way. So you are actually manipulating the abstract syntax tree almost directly instead of having to type in every single character of every single line. While other iPad solutions run code on an external server, Lisping runs locally using a Scheme interpreter written in C, called TinyScheme. Lisping has been written with usability in mind\,\cite{lisping}: 

``Textviews and virtual keyboards aren't the only option for coding on iOS\@. Editing source code character by character is a concept wedded to the keyboard and it is not necessarily the best option for a device with no keyboard.''

Lisping is also a touch-based programming editor with support for multiple types of gestures, but it is the fact that it is structured that makes it interesting to our study.

We are fairly experienced programmers with only little knowledge of the Scheme language and it's syntax, and we had a hard time understanding and using Lisping. The underlying TinyScheme interpreter most often gave us “Error Unknown” compiler messages even though we were writing examples straight from the Scheme website. Not being able to pinpoint what was syntactically wrong with the written code and recover from that error was a major problem, as stated by Nielsen's ninth heuristic\,\cite{nielsen1990heuristic}.

The editor supports a range of touch gestures that are not immediately obvious to the user. Low memorability is generally a problem with touch gestures as described by Nielsen\,\cite[p. 141]{nielsen2013mobile}. In Lisping, you have to open the Lisping guide in the upper right corner and read through 6 pages of a PDF document to familiarize yourself with these gestures. This is a memorability problem\,\cite{nielsen1990heuristic}.

As can be seen in Figure~\ref{fig:Lisping_screenshot}, there are also several buttons on the bottom of the UI\@. 
These are all icons and except for the delete and edit ones, they have no standard meaning in iOS or are used differently from the standard meaning. 
The undo button is e.g.\ a backwards-pointing triangle which could be interpreted to mean “Back” in iOS\@.
Given the fact that the user has no chance of remembering all these non-standard icons, they clearly violate the ``Recognition rather than recall'' and ``Consistency and standards'' usability guidelines\,\cite{nielsen1990heuristic}. 

\begin{figure}
	\centering
		\includegraphics[width=110mm]{diagrams/Lisping_screenshot.png}
	\caption{A cropped screenshot of the Lisping interface, where the bottom
	toolbar consists of a range of non-standard icons and the program is filled
	with [+] buttons}
\label{fig:Lisping_screenshot}
\end{figure}

Furthermore, these icons are all located close to each other and potentially a whole screen's length away from where the user's attention is supposed to be when programming --- that is, on the code. All but two of the actions that these icons perform can be triggered with gestures performed on the source code. We see this as a incomplete attempt on providing accelerators for the expert user.

The problem with these gestures is that they are uncomfortable and cumbersome to use. To select an expression to edit or run in the REPL, the user must ``reverse pinch'' the text. This is a non-standard way of highlighting text in iOS and it feels very awkward and unresponsive not to mention that it gets painful to do after three to four times. It is even straining to perform normal tap gestures due to the small size of all the expressions in the editor.

The final usability problem we discovered with Lisping was the amount of [+] buttons inlined in the code. These buttons are supposed to indicate that an expression can be added there, but what they also do is make the code much harder to read.
In Figure~\ref{fig:Lisping_screenshot} we have a fairly simple program with
only three [+] buttons. This becomes more severe when the program grows.

\paragraph{Takeaways}
\begin{itemize}
	\item \textbf{Ta-8}: The error messages from the compiler/interpreter should indicate where in the code the syntax errors have occurred. Simply presenting ``Unknown Error'' is frustrating for the user.
	\item \textbf{Ta-9}: It should not be necessary to add a PDF document with instructions to how the gestures and buttons work. It should be immediately recognized by the user because it is all presented according to the platform standards. This serves as a reminder that the overuse of clever gestures harms the user experience.
	\item \textbf{Ta-10}: Littering a structured editor with [+] (add) buttons, to indicate that expressions can be added at these positions, is not a good idea. While a good alternative for this solution is hard to come up with we should try to design a different way.
\end{itemize}

\subsubsection{Eastwest}
\label{subsub:Eastwest}
Even though Eastwest is not a touch-based editor, we have included it in this section as it is a structured editor. 
While it is more of an experiment than a fully featured editor, we find it very interesting. It is an editor for a functional language that allows the user to ``fill in holes'', i.e.\ fulfill goals of a program from a popup. The context appears right under the goal and thus works as a well-placed auto completion tool. We have been unable to complete a heuristics evaluation of Eastwest, as we were unable to install it on our computers. Several videos are available, though, and it is interesting to see how functional data types are being defined and functions are being built in a structural way.

\paragraph{Takeaways}
\begin{itemize}
	\item \textbf{Ta-11}: Having a popup close to the goal makes for quick and easy access. The better that popup is at guessing the right thing for the goal, the faster the tool will seem.
	\item \textbf{Ta-12}: Eastwest is a good indication that functional languages work well in a structured editor.
\end{itemize}


\subsubsection{TouchDevelop}
\label{subsub:TouchDevelop}
TouchDevelop allows developers to make touch and accelerometer enabled games for any device with a web browser by providing a web app. It is generally meant as a way of getting children interested in programming, and the language has been specially designed for this purpose and works hand in hand with the development platform to provide a good experience for touch devices.

TouchDevelop displays a large auto completion area along the bottom of the interface, which shows relevant suggestions from the context. 
This means that you rarely have to type characters into the editor. 
Variable names are automatically generated at the push of a button.
Figure\ref{fig:TouchDevelop_screenshot} shows a sample program drawing a
turtle and a simple triangle pattern.

\begin{figure}
	\centering
		\includegraphics[width=80mm]{diagrams/TouchDevelop_screenshot.png}
	\caption{A screenshot of the Turtle Triangle Spiral sample from
	TouchDevelop\,\cite{TouchDevelop:TurtleTriangleSpiral}}.
\label{fig:TouchDevelop_screenshot}
\end{figure}

The first thing you notice as an experienced programmer is how the TouchDevelop language looks like modern imperative languages like C\# and Java, and this recognizability enabled us to get started with the platform right away. 
To educate new users there is a long tutorial that guides you through your first game. 

Even though TouchDevelop works on all touch devices that has a web browser it is not really optimized for other gestures than the standard single tap gesture that works on all platforms.

To fit more code on the screen and create a better programming overview, TouchDevelop uses a rather small font for the code you write. 
When you tap the line of code you wish to edit, that statement is enlarged and you can relatively easily move the cursor around. 
Additionally, buttons appear to insert new lines above and below the selected segment. 
This design is minimalistic and controls only show up when you need them\,\cite{nielsen1990heuristic}.

The auto completion area that is presented for selecting method calls and objects from is convenient when it presents you with just what you needed, but in most cases you find yourself struggling to find just that.
In these cases one must search using the virtual keyboard.
When filling out a hole in the program, the editor does not seem to narrow down the possibilities, as it displays options that would result in obvious type errors.

The user experience of TouchDevelop suffers from the fact that it is a web app written with no specific platform in mind. The biggest annoyance is when you want to move around the cursor. This is clumsy on the iPad and often requires that you use buttons.

\paragraph{Takeaways}
\begin{itemize}
	\item \textbf{Ta-13}: To have a contextual overview can speed up the tasks of programming just as we saw with Eastwest.
	\item \textbf{Ta-14}: The type system of the TouchDevelop object oriented programming language is not ideal if you want to expose a context for the user to pick from. The type system simply can't narrow down the possibilities efficiently.
	\item \textbf{Ta-15}: Using a smaller font for code that is not in focus is a clever way to create a better overview for the user.
	\item \textbf{Ta-16}: There is a clear difference between the feel of a web app like TouchDevelop and a native app like Lisping.
\end{itemize}

\paragraph{}

As we have learned, it is paramount to have as little use of the virtual keyboard as possible, and adopting the structured approach together with the strong type system of Idris might be able to give us just that.

\subsection{Visual Programming Languages} % (fold)
\label{sub:visual_programming_languages}
In this section we will investigate three editors with visual components.
It is possible that a more visual approach will work well on a touch-based interface, so even though Idris is not a visual language at all, we might find elements from these editors that our applicable to our design.

\subsubsection{Labview}
\label{subsub:Labview}
Labview is a development environment featuring the G VPL which is designed for dataflow programming and uses the popular ``boxes and wires'' abstraction, with boxes performing various functions, and wires leading data between the boxes.
Labview also lets users combine simple widgets to build GUI programs, using the G language for logic.
It is widely used in the scientific community, especially to work with data collected from sensors, and is often referenced in articles on VPLs\,\cite{Green96usabilityanalysis,DBLP:journals/ijmms/PetreB99}.
As such, we find it to be a good representative of this style of VPL\@.
Figure~\ref{fig:LabViewGettingStarted} shows a simple example of calculating a
triangle's area.

\begin{figure}[h]
\centering
\includegraphics[width=110mm]{diagrams/LabView_screenshot.png}
\caption{A simple LabView program from the tutorial\,\cite{LabView:GettingStarted}}.
\label{fig:LabViewGettingStarted}
\end{figure}

While complex programs in G can be initially overwhelming to even experienced programmers, Whitley and Blackwell\,\cite{WHITLEY2001435} show that seasoned Labview programmers prefer the visual syntax to a textual one.

\paragraph{Takeaways}
\begin{itemize}
	\item \textbf{Ta-17}: One interpretation of Whitley and Blackwell's findings\,\cite{WHITLEY2001435} is that although Labview can be hard for programmers not used to the visual syntax, the higher learning curve pays off in the long run.
	\item \textbf{Ta-18}: The ``boxes and wires'' approach is good at indicating structure and could maybe be transfered to a more general purpose language.
\end{itemize}


\subsubsection{Scratch}
\label{subsub:Scratch}
Scratch is a VPL designed to be easy for beginners to use, facilitating the development of simple audio/visual programs. It is implemented as a web app, and features built-in tutorials, tips and help.
The language is built around a scene containing sprites. A scene can have multiple sprites, and each sprite can be governed by multiple scripts written in the Scratch visual language. The language itself consists of blocks of different shapes, colors and sizes. Scratch does not use the dataflow paradigm, as Labview does. Instead, one programs by snapping blocks together. As can be seen from Figure~\ref{fig:ScratchProgram}, the blocks have different shapes, and only complementing shapes can be put together. Color is used to describe different types of blocks.

\begin{figure}
	\centering
		\includegraphics[width=60mm]{diagrams/ScratchProgram.png}
	\caption{A simple Scratch program that moves and spins the sprite if the space button is pressed.}
\label{fig:ScratchProgram}
\end{figure}

A wide range of solutions based on or similar to Scratch have been made over the years\,\cite{hosick2014}, but one is, in particular, interesting to us as it is a port of Scratch to the Apple iPad. It is called Hopscotch\,\cite{hopscotch} and generally does exactly what Scratch does just by incorporating two types of simple touch gestures.

The most interesting part of Scratch is the visual language itself.
Use of different colors for different types of elements (control structures, data management, input), along with different shapes for showing which elements can be combined makes it very easy to discriminate between different components.
Symbols are used in a semantically immediate way in some places, such as an arrow pointing back to the start of a loop. 
Complexity management is a problem for Scratch, as there does not seem to be a way to define new functions; each script is composed of exactly one function, and scripts cannot access anything from other scripts.
This means functions can become very long and unwieldy.
Scalability in VPLs has been explored in depth by Green and Petre\,\cite{green1992visual}.

\paragraph{Takeaways}
\begin{itemize}
	\item \textbf{Ta-19}: When designing a visual syntax, making it visually obvious which elements go together greatly increases the ease of learning.
	\item \textbf{Ta-20}: Using colors and shapes to differentiate elements is very important in a visual language.
	\item \textbf{Ta-21} Complexity management is hard with visual languages\,\cite{green1992visual}.
	\item \textbf{Ta-22}: Use all the aspects of visual expressiveness when designing a visual language, e.g.\ shape, color, texture, position, etc.
\end{itemize}

\subsubsection{Epigram}
\label{subsub:Epigram}
The Epigram\,\cite{mcbride2005epigram} language is a functional and dependently typed language made by Conor McBride, which aimed to let the programmer create compiler certified proofs using intuitionistic logic.
It is not a traditional visual programming language, but it has been put in this category due to its 2D syntax.
Especially the syntax of the data type specification is interesting.
In Epigram, defining data is done in a more visual way than newer dependently typed languages. 
This is done by imitating the standard way of writing inference rules.
As can be seen in Figure~\ref{fig:epigram_data}, the definition of e.g.\ a data type is split into three lines and its 2D syntax almost resembles ASCII art. 

\begin{figure}[htbp]
	\centering
	
	\lstinputlisting[firstnumber=1,basicstyle=\ttfamily\scriptsize]{code/epigram_data}
	\caption{The natural numbers defined in Epigram}
\label{fig:epigram_data}
\end{figure}

While it has not been possible to get a version of Epigram working for any sort of test, we are interested in how this way of expressing data types could benefit the user in a more structured editor, that does not require the user to maintain ASCII art, but still displays the program in a more visual way.

\subsubsection{Takeaways}
\begin{itemize}
	\item \textbf{Ta-23}: The more visual way of writing data types could be more intuitive for people used to reading that sort of notation. This goes well with Nielsen’s ``Match between system and the real world''-heuristic.
\end{itemize}

\paragraph{}

While we think a more visual way of programming might be a good fit for a touch-based interface, the issues with complexity management and the high learning curve make us weary.
Instead of a completely visual editor, it might be possible to take certain elements to produce more readable programs. 
Especially the 2D data syntax from Epigram seems interesting.

% subsection visual_programming_languages (end)

\subsection{Existing Solutions Overview}
% Please add the following required packages to your document preamble:
% \usepackage{multirow}
% \usepackage[table,xcdraw]{xcolor}
% If you use beamer only pass "xcolor=table" option, i.e. \documentclass[xcolor=table]{beamer}

\begin{table}[ht]
{\renewcommand{\arraystretch}{2}%
\begin{tabularx}{\textwidth{}}{|c|X|c|c|c|}
\hline
	& \textbf{Description}
	& \textbf{Touch-based}                                              
	& \textbf{Visual syntax}                          
	& \textbf{Stuctured editor} 
\\ \hline
	\textbf{CodeToGo} & & \cellcolor[HTML]{B6D7A8}{\color[HTML]{9AFF99} }                  & & \\ \cline{1-1}
	\textbf{Textastic}  & \multirow{-2}{*}{
	Textual iPad env.
	} & \multirow{-2}{*}{\cellcolor[HTML]{B6D7A8}{\color[HTML]{9AFF99} }} & \multirow{-2}{*}{}                              & \multirow{-2}{*}{}                                                   
\\ \hline
	\textbf{Raskell}      & 
		Mobile Haskell & \cellcolor[HTML]{B6D7A8}{\color[HTML]{9AFF99} } & &                                                               
\\ \hline
	\textbf{Lisping}      & 
		Mobile Lisp & \cellcolor[HTML]{6AA84F} & & \cellcolor[HTML]{B6D7A8}

\\ \hline
	\textbf{Eastwest}     & 
		Functional programming system with structured editor
		 & & & \cellcolor[HTML]{274E13}
\\ \hline	
	\textbf{TouchDevelop} & 
		Imperative language running as a web app
		 & \cellcolor[HTML]{6AA84F} & & \cellcolor[HTML]{B6D7A8} 
\\ \hline		
	\textbf{Labview}      & 
		Visual data flow programming & & \cellcolor[HTML]{38761D} &
\\ \hline
	\textbf{Scratch}      & 
		Beginner-friendly programming by dragging boxes
		  & & \cellcolor[HTML]{274E13} & \cellcolor[HTML]{274E13}
\\ \hline
	\textbf{Epigram}      & 
		Functional with Dependent Types & & \cellcolor[HTML]{6AA84F} &
\\ \hline
\end{tabularx}
}
\caption{Existing solutions overview.}
\label{table:existing_solutions_overview}
\end{table}
In this section we have analyzed a range of existing solutions and for each reached a list of takeaways that we will use when determining the requirements for our design.

The current state of the art touch-based iPad apps either only use the single tap gesture, thus wasting a golden opportunity, or use several gestures without concern for the usability of them. 
The issue seems to be that they are all essentially manipulating text and that the standard text input UI elements on the iPad already support several gestures. 
If the app tries to incorporate too many gestures to manipulate the structure of these text fields there may be conflicts. 
Finding a solution to this problem is a major challenge.

We also discussed a few structured editors, where two of them support functional languages. 
Having the editor be cognizant of the underlying structure of the program seems to lessen the amount of typing the user has to perform on the virtual keyboard. 

Finally, a more visual approach might be appropriate, but several issues, such as complexity management issues, means special care must be taken.
Table~\ref{table:existing_solutions_overview} gives an overview of the existing solutions. 

%!TEX root = ../Touch Based Idris.tex
\section{Requirements}
\label{sec:Requirements}
Based on the initial analysis we will here present various sets of requirements for the new touch-based Idris syntax.

\subsection{Functional Requirements}
\label{subsec:FunctionalRequirements}
here goes all specific requirements that describes how a system should do something

\subsection{Usability Requirements}
More non-functional

How gestures should enhance the programming experience
Gestures are hard to discover (ref), so there should be a way to use the interface without the advanced gestures

\subsection{Defining Scope}
The aim of the project is to explore usability issues related to defining a standard data type with the touch-based interface as well as defining a function to manipulate this data type. We have chosen to focus on the canonical Vect data type, which is, as we explained in section\todo{insert ref}, often used as the basic example of how dependent types work. Furthermore we will be letting our usability test subjects define the zip function (also explained in section\todo{insert ref}). This function allows the testing of automatic initial clause creation, automatic case splitting, and rhs code inference.

Any language features of Idris that are not directly needed to be able to test the usability of defining the above-mentioned top-level declarations will thus be omitted. This means we will only not handle language features such as dependent pairs, explicit types, with notation \todo{insert more?}.
%!TEX root = ../Touch Based Idris.tex
\section{Initial Design}
\label{sec:InitialDesign}






% Mention somewhere: Low memorability with gestures (p 141) It’s probably not a good idea for us to
% invent too many new gestures, as they are hard to discover and people don’t
% remember them Alternatively gestures can be a power user functionality that
% makes it faster to work with the app. We should put in standard single tap
% ways of doing things in any case (p. 143)

\subsection{Usability Test}
\label{sec:UsabilityTest}
With a design in hand, it was to test it. The purpose of this test was to get
an initial idea of how our ideas worked before spending time implementing it.
We created paper cutouts based on our mock-ups to represent the different
elements of the interface, with a test facilitator manipulating the cutouts to
emulate an interactive environment. See Appendix \todo{Ref} for the full notes
from the test.

\subsubsection{Participants}
As it is beyond the scope of this project to teach new users about dependent
types, our test participants needed experience with a programming language
featuring dependent types, ideally Idris, prior to our test. This obviously
severely limited the number of readily available candidates. Because we wanted
to conduct later tests with subjects without knowledge of earlier iterations of
the design, we limited ourselves to two test subjects for this first test. An
overview of the test subjects can be see in Table \ref{table:first_test_subjects}.

\begin{table}[h]
\centering
\begin{tabular}{| l | l | p{5cm} | p{5cm} |}
\hline
Subject & Age & Occupation & Experience \\ \hline
\#1 & 24 & Masters student at IT-University studying programming languages & 6 months experience with Idris, several years of experience with functional languages in general \\ \hline
\#2 & 27 & Masters student at IT-University studying programming languages & 6 months experience with Idris, several years of experience with functional languages in general \\ \hline
\end{tabular}
\caption {Test subjects}
\label{table:first_test_subjects}
\end{table}

\subsubsection{Session Details}
The test was conducting in a meeting room at the IT University, with three
people present: The test subject, the test facilitator, and a note taker. The
tests were recorded. First, the test subjects were told about the project in
general. They were then told of the overall structure of the test: They would
be given two tasks, with three steps each. Each step would be explained when
they reached. They were asked to think aloud during the test, and were told to
ask if they felt they had been stuck for a long time. Periodically, we would
ask questions about what they were thinking.

\subsubsection{Tasks}
The two tasks they were asked to complete are listed in Figure \ref{figure:first_tasks}.
It should be noted that since the test was concerning the interface itself, and
not programming in Idris, the test subjects had access to the definitions for \texttt{Vect} and \texttt{zip} textual Idris. See Section \ref{subsec:Idris}
for more on \texttt{Vect} and \texttt{zip}.

\begin{figure}
\centering
\begin{itemize}
	\item \textbf{Task 1}: Define a data declaration for the vector type.
	\begin{itemize}
		\item \texttt{T1.1}: Specify an identifier for the type (\texttt{Vect}), along with its type (\texttt{Nat -> Type -> Type})
		\item \texttt{T1.2}: Specify the Nil constructor (\texttt{Nil: Vect z a})
		\item \texttt{T1.3}: Specify the Cons constructor (\texttt{(::): a -> Vect k a -> Vect (S k) a})
	\end{itemize}
	\item \textbf{Task 2}: Define the zip function for vector type.
	\begin{itemize}
		\item \texttt{T2.1}: Specify the identifier for the function (\texttt{zip}), along with its type (\texttt{Vect k a -> Vect k b -> Vect k (a, b)})
		\item \texttt{T2.2}: Specify the first case (\texttt{zip Nil Nil = Nil})
		\item \texttt{T2.3}: Specify the second case (\texttt{zip x::xs y::s = (x, y) :: zip xs ys})
	\end{itemize}
\end{itemize}
\caption{Tasks for the first usability. The text in parentheses are what we considered the correct answer, and was not given to the test subjects.}
\label{figure:first_tasks}
\end{figure}

\subsubsection{Issues}
\label{sec:first_issues}
As was expected, several issues were encountered. It was quickly apparent that
the data declarations were the most problematic aspect of the design, and that
we would need to revisit their design. Task 2 on the other hand went quite
smoothly in comparison, and both subjects were impressed by the degree to
which Idris could save them from typing. We have listed the issues we
encountered below. \\ \\
\textbf{I1: Data declarations}.
The main issues our test subjects experienced had to do with the data
declarations. While both subjects had seen this way of writing types
before, it was not immediately clear to them how they should be filled out.
They did eventually get through it, but they required a great deal of help
from the test facilitator. Especially step T1.1 was difficult.\\ \\
\textbf{I2: Suggestions for parameterized types}.
When filling in a type, a popover with suggestions appear. In our mock-up,
these suggestions were always correct. But both subjects wondered how one would
fill in types that take multiple parameters. \\ \\
\textbf{I3: Gesture overload}
Both subjects wondered how to differentiate gestures for editing versus
gestures for case splitting or autocompletion. For example, double tapping on
text already has an action associated with it. \\ \\
\textbf{I4: Managing contexts}
After having finished step T1.3, both subjects had trouble leaving the editing
context. Only after a great deal of experimentation did they manage to leave
it.

\subsubsection{Recommendations}
\label{sec:first_recommendations}
Since issue I1 was by far the most problematic issue we discovered, we have
focused on mitigating it in our recommendations. The basic strategy is to make
it easier to understand through examples and help in the interface. If the
changes listed below are not enough, it might be necessary to completely
redesign the data declarations, perhaps by modeling them more closely after how
they look in textual Idris.

\begin{itemize}
	\item \textbf{Re1} (I1): In the text field for inputting the name and type for Data, insert an initial colon (``\texttt{:}''), separating the identifier from the type. This might make it easier to see what goes where.
	\item \textbf{Re2} (I1): Have initial hint text in text input fields, in a light color, which disappears when the fields gain focus. E.g. in the text field for the identifier, it could say ``Identifier'' or ``Name'', while the input field for the type can say ``Type''.
	\item \textbf{Re3} (I1): Differentiate borders around text fields more, to make it clear which fields have been filled, which must be filled, and which might be filled.
	\item \textbf{Re4} (I1): Show the definition for \texttt{Nat} to begin with, so the user can see an example.
	\item \textbf{Re5} (I2): Do not try to fill parameters for types, as we cannot know what they will be. Instead, just leave blank spaces so the user can fill them in.
	\item \textbf{Re6} (I3): To avoid conflict between editing and case splitting/autocompleting, implement a new gesture for ``auto'', perhaps swiping down on a term.
	\item \textbf{Re7} (I4): To avoid problems switching between contexts, we can remove the different contexts. This removes complexity from the interface, although it might make it harder to get a good overview of your program, as less can fit on a single screen.
\end{itemize}

%!TEX root = ../Touch Based Idris.tex 
\chapter{Architecture}
\label{sec:Architecture}

We decided to write our app for the Apple iPad, as it is the most prolific
tablet today. We also want to be able to use the various interactive features
available in the Emacs mode for Idris, such as case
splitting and auto completion (F-1, F-2, F-3). One way to achieve this would be to implement
the required functionality directly in our app. But there is a much easier way.
The Emacs mode simply communicates with the Idris interpreter over a socket. 
In this way the Emacs mode can support advanced features that require an
understanding of the underlying semantics of a piece of code, while letting
Idris do all the hard work. We would like to use the same mechanism, but there
are some obstacles. First, we would have to compile Idris for iOS, and make 
sure it works. The Glasgow Haskell Compiler \emph{does} support 
cross-compiling for iOS\,\cite{ghc_ios_crosscompiler}, but the process is not very mature, and
getting Idris (and all of its dependencies) to run on iOS represents an 
unknown amount of work. Second, Apple has very strict rules for interpreting
code on iOS\@. This might make it very difficult to get the app distributed on
the App Store, the only official way to publicly distribute iOS apps.

Because of these factors, we decided to let the Idris interpreter run on a 
host computer, with a thin client running on the iPad, and a server program 
facilitating communication between the client and Idris.

\subsubsection{Abstract Syntax}
As described in the scope section (\ref{sec:defining_scope}), we will only
provide support for a subset of the total language constructs in our prototype.
The definition of the AST of the IdrisTouch solution can be found in Appendix
\ref{chap:IdrisTouch_AST}.


\section{Communication}
There are two layers of communication in this setup: one between the client and
the server, and one between the server and Idris. Several different strategies
exist for each layer. 

\subsection{Server -- Idris}
The Emacs mode for Idris communicates to the Idris slave via a simple protocol. This can be avoided by writing the server program in
Haskell, and using Idris as a library instead. This way we can call the
functions we need directly. We still have a problem though, as Idris expects
concrete textual Idris. One solution would involve transforming the AST from 
the client to concrete textual Idris, either on the server or the client. When
receiving replies from Idris, we would then have to parse textual Idris to the
client's AST\@, and send this back. An alternative is to transform the 
client's AST directly to high-level Idris syntax, and bypassing the parsing
steps in Idris, so the functions we need from Idris accept Idris AST directly.
This way we can skip the generation and parsing of concrete Idris. In the end,
we chose this solution, as it seemed cleaner, although it meant learning much
about how Idris is implemented in Haskell.
\todo{Insert diagram showing the two possibilities}

\subsection{Server -- Client}
This layer is somewhat less difficult. The server exposes a simple RESTful API 
over HTTP\@. This makes it easy for the client and server to communicate across
different networks. When transferring a program from client to server, we
serialize the client AST to JSON\@.

\section{Server}
The architecture of the server is very simple, and consists of two main 
layers. The outer layers uses the Snap Framework\,\cite{snap_framework} to implement the 
web-services. This layer also handles marshaling to and from JSON\@, which is 
implemented using the Aeson library\,\cite{aeson_package}. The inner layer handles the 
transformations to and from the Idris high level syntax, and performs the 
actual operations, such as case splitting.

\section{Client}
The iPad client is, of course, written in Objective-C as is the standard. The
software architecture is based on the Model View ViewModel (MVVM) design 
pattern\,\cite{JohnGossman:MVVM}, which is an extension of the popular Model View Controller 
(MVC) pattern. MVVM nicely separates business logic from the view-related
display logic, and the latter is stored in the ViewModel object to unclutter 
the controller object, which in turn only contains business logic.

To handle asynchronous network calls as well as input from the user we have
used the ReactiveCocoa (RAC) Framework\,\cite{reactiveCocoa}, which is an Objective-C framework for
Functional Reactive Programming\,\cite{Furrow:FunctionalReactiveProgramming}. In RAC almost everything is a
signal. Signals follow the Future/Promise design pattern and provide an
efficient way of handling future events.

The model hierarchy resembles the abstract syntax from Section \ref{subsec:AbstractSyntax},
where the type constructors are represented as abstract classes on the client
and the specific constructors, concrete classes. Note that abstract classes in
Objective-C are only abstract by convention, as the language does not directly
support this feature.

It is actually not only the model hierarchy but also the view hierarchy that closely resembles the AST. Figure~\ref{fig:clientViewArchitecture} shows a simplified class diagram of the UI classes of the client. This overview gives an idea of
the resemblance to the abstract syntax. As with the abstract syntax there are
two ``levels'' of views --- top level and input level. The MainView holds a
number of top level declarations, that each hold a number of input views. The
InferenceRuleView is used by the DataDeclarationView only.

A GroupInputView contains one or more AbstractInputViews, and they can be specialized with regards to appearance and behavior. This way of building
the interface makes it easy to wire input views to the underlying model object
hierarchy representing the AST using ReactiveCocoa, as it is the same type of nesting we have with terms. 

\begin{figure}
	\centering
		\includegraphics[width=110mm]{diagrams/client_side_class_diagram.pdf}
	\caption{A subset of the client-side diagram showing how the view hierarchy
	almost mirrors the Abstract Syntax tree.}
\label{fig:clientViewArchitecture}
\end{figure}












%!TEX root = ../Touch Based Idris.tex
\section{Implementation}
\label{sec:Implementation}

Having the architecture marked down, we could focus on another iteration of
usability testing. It was the goal to create a prototype running on the iPad,
ready for usability testing with a bigger group of participants than in the
mock-up phase.

\begin{figure}
	\centering
		\includegraphics[width=110mm]{diagrams/ipad_interface.PNG}
	\caption{The initial iPad interface.}
\label{fig:initialiPadInterface}
\end{figure}

\begin{figure}
	\centering
		\includegraphics[width=70mm]{diagrams/data_declaration.png}
	\caption{A data declaration before the user has filled out anything.}
\label{fig:data_declaration}
\end{figure}

To produce this prototype we had to follow the usability recommendations of
Section~\ref{sec:first_recommendations}	while still keeping the goals and requirements of
Section~\ref{sec:GoalsAndRequirements} in mind. Figure~\ref{fig:initialiPadInterface}
shows an example data type as well as the zip function defined in the visual,
high-level syntax. It is evident that this is still a prototype, but the reader
should be able to get an idea of how it works.

One of the things that were not implemented in the initial prototype was the
TouchDevelop-like focus shifting, where different top-level elements would be
in focus at different times. In this prototype all elements are in focus all
the time.

The data declarations (see Figure~\ref{fig:data_declaration}) were kept the same as in the mock-up except for the
placeholder text and colon indicating the separation between the identifier and
type.

This prototype consists of a lot of custom elements, and the way that they work
together has been challenging to implement. Figure~\ref{fig:clientViewArchitecture} shows a simplified class diagram of the
client. It shows the UI classes of the project to give an idea of
the resemblance to the abstract syntax. As with the abstract syntax there are
two ``levels'' of views --- top level and input level. The MainView holds a
number of top level declarations, that each hold a number of input views. The
InferenceRuleView is used by the DataDeclarationView only.

A GroupInputView contains one or more AbstractInputViews. This way of building
the interface makes it easy to wire input views to the underlying model object
hierarchy, as it is the same type of nesting we have with terms. \todo{Explain
what the group input view is and how it can be customized}

\begin{figure}
	\centering
		\includegraphics[width=110mm]{diagrams/client_side_class_diagram.pdf}
	\caption{A subset of the client-side diagram showing how the view hierarchy
	almost mirrors the Abstract Syntax tree.}
\label{fig:clientViewArchitecture}
\end{figure}

\subsection{Second Usability Iteration}
\label{sec:SecondUsabilityTest}
The greatest change in the second usability test was the fact that it was
performed on an actual iPad, using a prototype interface. The full report is
included in Appendix \todo{insert ref}. As in the first test, S2.4 refers to
the fourth point in the test summary for subject 2.

\subsubsection{Participants}
For our second usability test, we used two subjects from the first test,
subject \#1 and \#2. Their input was interesting, as they already had some
experience with the ideas in our representation from the mock-up test.
Hopefully this would let them focus more on the interactions with interface,
and less on learning a new way of presenting data. Subjects \#3 and \#4 had
never seen the interface before, and as such represented totally new users.
Their feedback was also very useful, as this iteration tried to make the first
time experience less frustrating.

\begin{table}[h]
\centering
\begin{tabular}{| l | l | p{5cm} | p{5cm} |}
\hline
Subject & Age & Occupation & Experience \\ \hline
\#1 & 24 & Masters student at IT-University studying programming languages & 7 months experience with Idris, several years of experience with functional languages in general \\ \hline
\#2 & 27 & Masters student at IT-University studying programming languages & 7 months experience with Idris, several years of experience with functional languages in general \\ \hline
\#3 & 27 & Masters student at IT-University studying programming languages & Very little experience with Idris. 1 year experience with Coq. Several years of experience with functional languages in general \\ \hline
\#4 & 23 & Masters student at IT-University studying programming languages & 6 months experience with Idris. Several years of experience with functional languages in general \\ \hline
\end{tabular}
\caption{Test subjects}
\label{table:second_test_subjects}
\end{table}

\subsubsection{Session Details}
Like the first test, this test was conducted in a meeting room at the IT
University, with three people present: The test subject, the test facilitator,
and a note taker. The tests were recorded. This time the test consisted of four
tasks, with the first task designed to get the subjects acquainted with the
data declarations. The next two tasks were identical to the first test, the
definition of the \texttt{Vect} data type and the \texttt{zip} function for
\texttt{Vect}. The final task concerned the manipulation of order of
declarations in the program they defined. As the app was still at prototype
stage, several bugs occurred during the subjects' use of the app. In these
cases the nature of the bug was quickly explained, and the facilitator
explained how to work around the issue.

\subsubsection{Tasks}
The four tasks they were asked to complete are listed in 
Figure~\ref{figure:second_tasks}.
As in the first test, the test subjects had access to the definitions for
\texttt{Vect} and \texttt{zip} textual Idris. See Section~\ref{subsec:Idris}
for more on \texttt{Vect} and \texttt{zip}.

\begin{figure}
\centering
\begin{itemize}
	\item \textbf{Task 1}: The user is shown the \texttt{Nat} and \texttt{Fin} data declaration in the program.
	\begin{itemize}
		\item \textbf{T1.1}: Describe the \texttt{Nat} type.
		\item \textbf{T2.2}: Describe the \texttt{Fin} type.
	\end{itemize}
	\item \textbf{Task 2}: Define a data declaration for the vector type.
	\begin{itemize}
		\item \textbf{T2.1}: Specify an identifier for the type (\texttt{Vect}), along with its type (\texttt{Nat -> Type -> Type})
		\item \textbf{T2.2}: Specify the Nil constructor (\texttt{Nil: Vect z a})
		\item \textbf{T2.3}: Specify the Cons constructor (\texttt{(::): a -> Vect k a -> Vect (S k) a})
	\end{itemize}
	\item \textbf{Task 3}: Define the zip function for vector type.
	\begin{itemize}
		\item \textbf{T3.1}: Specify the identifier for the function (\texttt{zip}), along with its type (\texttt{Vect k a -> Vect k b -> Vect k (a, b)})
		\item \textbf{T3.2}: Specify the first case (\texttt{zip Nil Nil = Nil})
		\item \textbf{T3.3}: Specify the second case (\texttt{zip x::xs y::s = (x, y) :: zip xs ys})
	\end{itemize}
	\item \textbf{Task 4}
	\begin{itemize}
		\item \textbf{T4.1}: Move the \texttt{Vect} declaration up below \texttt{Nat}.
	\end{itemize}
\end{itemize}
\caption{Tasks for the second usability test. The text in parentheses are what we considered the correct answer, and was not given to the test subjects.}
\label{figure:second_tasks}
\end{figure}

\subsubsection{Issues}
\label{sec:second_issues}
Many new issues were discovered in this usability test. This is not
surprising. In the first test, the facilitator took over the iPad's role, by
manipulating the mockups. This masked many flow and ease of use issues, which
have now been identified. During the test, several bugs were encountered. These
are not listed below, as they are not inherent to the design, but rather
results of the prototypical nature of the app.
\\ \\
\textbf{I1: Too many input fields}.
Most of the users reported that all the grey, unfilled input fields were
confusing. (S1.20, S2.2, S3.29b)
\todo{Insert screenshot}
\\ \\
\textbf{I2: Writing data declaration}
Everyone was able to decipher the data declarations almost immediately, but
all but one had trouble writing the \texttt{Vect} data declaration. Especially
the distinction between the premise area and the conclusion area seemed
problematic. (S1.1--1, S3.11--18, S4.3--7)
\\ \\
\textbf{I3: Arrows in data confusing}
The use of arrows in the premise area is confusing, as it implies a function. 
(S4.4)
\\ \\
\textbf{I4: Suggestions for parameterized types}
This issue persists from the original design. The subjects found it irritating
that after choosing a suggestion, they would have to manually change the
parameters using the standard iOS text editing facilities. Multiple subjects
mentioned that it broke their flow. (S1.10, S2.7, S3.15)
\\ \\
\textbf{I5: Lack of auto-closing parentheses and quotation marks}
We observed that users spent too much time navigating the virtual keyboards to
close pairs of symbols, when this can be done automatically.
\\ \\
\textbf{I6: Common symbols inaccessible}
In a similar vein, we noticed that some symbols commonly used when programming,
e.g. ``<'', took too long to find in the virtual keyboard.
\\ \\
\textbf{I7: Poor flow between input fields}
To move from one input field to the next, the subjects must manually tap each
input field. This broke the subjects flow. (S1.21, S3.30c)
\\ \\
\textbf{I8: Program can look cluttered}
After finishing all the tasks, the subject's program started to look cluttered.
\todo{Insert screenshot} (S3.29a)
\\ \\
\textbf{I9: Few definitions visible}
Related to Issue I8, definitions are quite large, so only a few are visible at
any time. This can make it hard for the subject to get an overview of their
work.
\\ \\
\textbf{I10: Unclear how to specify implicits and identifiers for arguments}
Some parts of the syntax have not been determined. Some of the subjects
wondered how one would make an argument implicit. (S1.9, S4.13)

\subsubsection{Recommendations}
\label{sec:second_recommendations}
As this represents the final iteration during this project, these
recommendations serve more as a reflection over how these issues might be
addressed.\todo{Ref to section with more reflection}

\begin{itemize}
    \item \textbf{Re1} (I1): Differentiate input fields to a greater degree. Perhaps outline required input fields in red, to differentiate them from other input fields.
    \item \textbf{Re2} (I1): Do not show input fields for declarations that don’t have focus.
    \item \textbf{Re3} (I1): When centering text that includes an extra input field, center the text without the input field. \todo{Insert screenshot}
    \item \textbf{Re4} (I3): Use semicolons instead of arrows.
    \item \textbf{Re5} (I4): When showing types that take parameters, indicate these parameters in the popup. Then, after choosing one, create two new sub input fields, with their own autosuggestions. Make sure theres a good flow between these fields.
    \item \textbf{Re6} (I5): Automatically close pairs of symbols.
    \item \textbf{Re7} (I6): Include a (customizable) shortcut bar for symbols.
    \item \textbf{Re8} (I7): After filling a field, automatically move input to next field. Perhaps include a button to move back and forth between fields.
    \item \textbf{Re9} (I8): Include more spacing between fields.
    \item \textbf{Re10} (I9): Include a cheat sheet on the right hand side, showing constructors and types for functions.
    \item \textbf{Re11}: In general, it might be good to separate text editing from general manipulation of the program. Text editing could occur in popups or a dedicated text input field, freeing gestures for other uses when manipulating the structure of the program.
\end{itemize}

\todo{sub section conclusion}

\subsection{Further Usability Iterations}
Provide a plan for further development of the prototype


%!TEX root = ../Touch Based Idris.tex
\section{Evaluation}
\label{sec:Evaluation}

%!TEX root = ../Touch Based Idris.tex
\section{Reflection}
\label{sec:Reflection}

\subsection{Process}


\todo{Mention that the original plan was to do a lot of Haskell/Idris hacking
but that it soon turned out that we had to focus on usability to find out what
made sense}

\subsection{Constraining factors}
During this project we have encountered an incredibly high amount of unforeseen constraining factors. Some of the simplest things turned out to require demanding workarounds due to the strict rules of the iOS platform and the vast amount of technologies and languages that had to work in combination in the finished product.

A good example of the unforeseen complexity of the simplest of components was writing a JSON parser on the client side. Most often this is a tedious task that can be left to standard or third party libraries, which was also the case on the server side in Haskell. A data definition with $n$ constructors in Haskell corresponds to one abstract class with $n$ concrete classes inheriting from the abstract class in Objective-C. It turns out that none of the top Objective-C JSON libraries had a good way of handling this abstraction, which forced us to spend valuable time extending a JSON library when we could have spent it on developing the UI further.

Another element that turned out to be way more time consuming than initially planned was defining and laying out the user interface for the prototype. Early on we chose to use Apple's Autolayout technology in which you define the relations between UI components just as you would on a white board instead of calculating spacings and offsets on a pixel level. The idea is that you define your interface once for all screen sizes and interface orientations, but in our case Autolayout turned out to bring just as many problems as benefits to the table, and we found ourselves spending way too much time on debugging our very custom interface.

\subsection{A Different Approach}
One of the underlying assumptions of this project is that it is not realistic to be able to program with the same speed and accuracy on a touch interface as you would with a normal keyboard. The most advanced touch-based editor is not nearly as powerful as the most advanced desktop editor/IDE. Keyboard-based editors have been developed for centuries whereas touch gesture-based editors have not been explored to the same extent. 

In this project we have taken an existing textual programming language, originally intended for desktop programming and tested how far we were able to take it usability wise. The question we ask is if it is this is the right approach. Keyboard-based editors have been developed with textual programming languages in mind, so why not develop a touch-based editor with a touch-based language in mind? Such a language does not, to our knowledge, exist today. It would require developers to completely rethink the way they program and exclude any and all use of the virtual keyboard.

Imagine a solution where the programmer would only use the fastest types of touch-gestures such as swipe, tap, and double-tap, where the hands would stay in the same positions to increase programming speed. Identifiers would be specified by using the built-in microphone. One should only support a very simple expression language to begin with and only try to support more complex constructs when this core had been tested to work as efficiently as keyboard-based programming.


\input{sections/10.Conclusion.tex}
\input{sections/11.References.tex}
\newpage
\input{sections/12.Appendix.tex}

\end{document}
