%!TEX root = ../Touch Based Idris.tex
\section{Background}
\label{sec:Background}
\subsection{Dependent Types}
In many programming languages, such as Java or Haskell, a type can depend on another type, such as \texttt{ArrayList<A>} in Java, or \texttt{[a]} in Haskell. By suppling a concrete type, we can say that a collection only holds this specific type of values, e.g. \texttt{ArrayList<String>} or \texttt{[String]}. But in a programming language with dependent types, it is possible to go further. With dependent types, a type can be dependent on \emph{values} as well as types. This allows for increased expressiveness in the type system, which can be used for ensuring correctness, conducting proofs, and improving tool support. For this project, it is the last point we will focus on. Dependent types have been implemented in several programming languages, such as Coq\todo{Ref}, Agda\todo{Ref}, and Idris, which we will examine next.

\subsection{Idris}
\label{subsec:Idris}
Idris is a dependently typed programming language, initially developed by Edwin Brady~\cite{Idris}. The syntax is heavily inspired by Haskell. Let us start by looking at a simple example. In Figure~\ref{fig:nat}, a data type representing the natural numbers is defined. It consists of two constructs \texttt{Z}, representing zero, and \texttt{S~n}, representing the successor of the natural number \texttt{n}.

\begin{figure}
\begin{alltt}
data Nat : Type where
  Z : Nat
  S : Nat \(\to\) Nat
\end{alltt}
\caption{The natural number data type.}
\label{fig:nat}
\end{figure}

The definition for \texttt{Nat} looks similar to how it would look in Haskell. Let us move to a slightly more complex example, where we make use of dependent types. In Figure~\ref{fig:vect}, the dependent vector, \texttt{Vect: Nat $\to$ Type $\to$ Type} is defined. This type takes two arguments: a \texttt{Nat}, \texttt{n}, describing the length of the vector, and a \texttt{Type}, \texttt{a}, specifying the type of the elements it contains.

\begin{figure}
\begin{alltt}
data Vect : Nat \(\to\) Type \(\to\) Type where
  Nil  : Vect Z a
  (::) : a \(\to\) Vect k a \(\to\) Vect (S k) a
\end{alltt}
\caption{The vector data type.}
\label{fig:vect}
\end{figure}

The first construct, \texttt{Nil}, tells us that the empty vector always has a length of \texttt{Z}, or zero. The second constructor, \texttt{(::)} or cons, takes a new element of type \texttt{a}, and prepends it to a vector holding elements of type \texttt{a}. The length of the resulting vector is increased by one compared to the previous vector. We say the vector is parameterized by the type \texttt{a}, as it is the same in both constructors. However, the length of the list, specified by the natural number \texttt{n}, is different in each constructor, so we say the vectors is a family of data types \texttt{indexed} by the natural numbers.

But why record the length of the vector in its type? How does this help us? A simple example of how this can be useful can be seen in the \texttt{zip} function, shown in Figure~\ref{fig:zip}.

\begin{figure}
\begin{alltt}
zip : Vect n a \(\to\) Vect n b \(\to\) Vect n (a, b)
zip Nil       Nil       = Nil
zip (x :: xs) (y :: ys) = (x, y) :: zip xs ys
\end{alltt}
\caption{Zip function for vectors in Idris.}
\label{fig:zip}
\end{figure}

The \texttt{zip} function takes two arguments. Each argument is a vector of the same length \texttt{n}. \texttt{n}, \texttt{a}, and \texttt{b} are inferred implicitly from their context. These arguments could also be explicitly declared as implicit by surrounding them with curly braces: \texttt{\{n:~Nat\}~$\to$ \{a:~Type\}~$\to$ \{b:~Type\}~$\to$ Vect~n~a~$\to$ Vect~n~b~$\to$ Vect~n~(a,b)} The function zips these two vectors together, producing a vector of the same length, containing tuples \texttt{(a, b)} of the values from the input vectors. If one tries to compile a program that attempts to zip two vectors of different lengths, the type checker will report it as an error, as opposed to failing at runtime. Notice how it is unnecessary to match on the cases where one vector has elements while the other is empty. This situation is impossible, due to the types. In fact, matching on this case is ill-typed, as \texttt{Z} and \texttt{S~n} cannot unify.\todo{Should we mention dependent pairs and the with-rule, when we don't support them?} For our final example, let us look at the \texttt{filter} function in Figure\ref{fig:filter}.

\begin{figure}
\begin{alltt}
filter : (a \(\to\) Bool) \(\to\) Vect n a \(\to\) (p ** Vect p a)
filter p [] = (_ ** [])
filter p (x::xs) with (filter p xs)
  | (_ ** tail) =
    if p x then (_ ** x::tail) else (_ ** tail)
\end{alltt}
\caption{Filter function for vectors in Idris.}
\label{fig:filter}
\end{figure}

The \texttt{filter} function uses two constructs that we have not seen so far, dependent pairs and the with rule. Dependent pairs let us specify a type that is dependent on another value. In \texttt{filter}, we do not know beforehand what length the resulting vector will have, so instead we return a dependent pair, where the first element is the length, and the second element is a \texttt{Vect} of that length. Not knowing the length is also a problem in the cons case of \texttt{filter}. If the head of the \texttt{Vect} passes the predicate \texttt{p}, we want to cons the tail of the \texttt{Vect} onto the head. But since we do not know what the length of \texttt{filter p xs} is beforehand (as it depends on how many elements passes the predicate \texttt{p}), we cannot construct the type for the resulting vector. But by using the \texttt{with} keyword, we can pattern match on the result of \texttt{filter p xs}. We know the length of \texttt{tail}, since we have filtered it already, so we can construct the resulting \texttt{Vect} without any problems. The underscores (eg. \texttt{(\_ ** tail)}) tell Idris to infer the value.