%!TEX root = ../Touch Based Idris.tex
\chapter{Evaluation}
\label{sec:Evaluation}
In this chapter we will evaluate our final design against the goals and requirements listed in Chapter~\ref{sec:GoalsAndRequirements}.
Fulfilling the functional requirements and usability requirements is a
prerequisite to fulfilling our goals, so we will evaluate those first.

\section{Requirements}
\begin{itemize}

\item \textbf{U-1}: Program structure is clearly communicated through the data and function declarations described in Chapter~\ref{sec:third_design_iteration}.
Only one of four test participants of the second usability iteration had minor
issues figuring out the program structure (Appendix~\ref{chap:SecondUsabilityTest},
S1.1-3, S2.1-3, S3.1-7, S4.1-2).
\item \textbf{U-2}: We use few, non overlapping touch gestures to accomplish specific goals,
but it should be usability tested as described in Chapter~\ref{subsec:third_usability_test}.
We are confident that our combination of touch gestures will work better than
what we achieved in the initial iPad prototype.
\item \textbf{U-3}: By explicitly giving focus to one top-level declaration at
a time we can ensure that the visual representation of the program scales just
as well as a purely textual program would. Top-level declarations not in focus
will not take up more screen real-estate than in a free-text editor. This
behavior is described in Section~\ref{subsec:managing_program_structure} and
should be usability tested in the third usability iteration.
\item \textbf{U-4}: We include several accelerators, such as reverse-pinch to create new declarations, that do not impede the novice user, and thus meet this requirement.
Another feature worth mentioning is the auto completion feature of the editor,
which should speed up the user's programming speed considerable making for a
faster edit-compile-test cycle.
\item \textbf{U-5}: Syntactical errors are prevented through the use of structured editing, and semantic errors are handled in a simple way, showing the user where the error is directly in the program.
This was described in Section\,\ref{subsec:error_handling}.
\item \textbf{U-6}: By only showing input areas when the current declaration is being edited, we keep clutter to a minimum, while still making it clear where the user can make changes. For example, compare the Lisping interface in Figure~\ref{fig:Lisping_screenshot} and our initial prototype interface in  Figure~\ref{fig:clutteredInterface} with our final design in Figure~\ref{fig:new_design_data_in_focus}.
\item \textbf{F-1}, \textbf{F-2}, \textbf{F-3}: Initial pattern matching, case splitting of pattern variables, and automatic metavariable solving is supported, and described in Section~\ref{subsec:new_design_function_dec}.
\item \textbf{F-4}, \textbf{F-5}: The user can add, delete, and edit data declarations, as described in Section~\ref{subsec:managing_program_structure} and Section~\ref{subsec:new_design_data_dec}
and function declarations, as described in Section~\ref{subsec:managing_program_structure} and Section~\ref{subsec:new_design_function_dec}.
\item \textbf{F-6}: By letting the Idris compiler run on a host computer (see Chapter~\ref{sec:Architecture}), we believe we do not break any App Store guidelines. However, in the end this is up to the App Store review team, and we cannot know without actually submitting an app.
\item \textbf{F-7}: The user can get an overview of what is available through the context popover, described in Section~\ref{subsec:new_input_model}
and depicted in Figure~\ref{fig:new_design_popover}.
\item \textbf{F-8}: Undo support is provided by buttons in the interface (see Chapter~\ref{sec:InitialDesign}), which we believe improves the undo-experience considerably
compared to shaking the device.
\item \textbf{F-9}: This requirement is met through the focus system described in Section
\ref{subsec:managing_program_structure}.
\item \textbf{F-10}: This requirement is met by the context-dependent virtual keyboard described in Section~\ref{subsec:virtual_keyboard}.

\end{itemize}

\section{Goals}
Our goal was to design an editor that leverages Idris to provide a usable solution for touch based editing, which delivers a competitive edit-compile-test cycle, with minimal use of the virtual keyboard, while remaining eligible for the Apple App Store.
We believe our latest design will make the fulfillment of these goals possible. 
Since the latest design was not tested with users, we can only guess at how effective
it would be at fulfilling G-2 and G-3. The edit-compile-test cycle has potential to be a lot faster than existing solutions due to the minimal reliance
on the virtual keyboard and the heavy reliance on auto completion. On the other
hand we have not yet tested this flow due to technical reasons, and we do have
a longer compiler response time, as the compiler runs on a remote server
instead of locally on the device. It is our hope that the advanced auto
completion feature will make up for the limitations of the distributed
architecture. This architecture was necessary to fulfill G-4.
We are convinced that with further iterations and more usability tests, our design could result in a product that is better suited for touch-based programming than existing solutions, but much work still lies ahead.
