%!TEX root = ../Touch Based Idris.tex
\section{Implementation}
\label{sec:Implementation}

\subsection{Second Usability Test}
\label{sec:SecondUsabilityTest}
The greatest change in the second usability test was the fact that it was
performed on an actual iPad, using a prototype interface. \todo{Write more
intro}

\subsubsection{Participants}
For our second usability test, we used two subjects from the first test,
subject \#1 and \#2. Their input was interesting, as they already had some
experience with the ideas in our representation from the mock-up test.
Hopefully this would let them focus more on the interactions with interface,
and less on learning a new way of presenting data. Subjects \#3 and \#4 had
never seen the interface before, and as such represented totally new users.
Their feedback was also very useful, as this iteration tried to make the first
time experience less frustrating.

\begin{table}[h]
\centering
\begin{tabular}{| l | l | p{5cm} | p{5cm} |}
\hline
Subject & Age & Occupation & Experience \\ \hline
\#1 & 24 & Masters student at IT-University studying programming languages & 6 months experience with Idris, several years of experience with functional languages in general \\ \hline
\#2 & 27 & Masters student at IT-University studying programming languages & 6 months experience with Idris, several years of experience with functional languages in general \\ \hline
\#3 & 27 & Masters student at IT-University studying programming languages & Very little experience with Idris. 1 year experience with Coq. Several years of experience with functional languages in general \\ \hline
\#4 & 23 & Masters student at IT-University studying programming languages & 6 months experience with Idris. Several years of experience with functional languages in general \\ \hline
\end{tabular}
\caption {Test subjects}
\label{table:second_test_subjects}
\end{table}

\subsubsection{Session Details}
Like the first test, this test was conducted in a meeting room at the IT
University, with three people present: The test subject, the test facilitator,
and a note taker. The tests were recorded. This time the test consisted of four
tasks, with the first task designed to get the subjects acquainted with the
data declarations. The next two tasks were identical to the first test, the
definition of the \texttt{Vect} data type and the \texttt{zip} function for
\texttt{Vect}. The final task concerned the manipulation of order of
declarations in the program they defined. As the app was still at prototype
stage, several bugs occurred during the subjects' use of the app. In these
cases the nature of the bug was quickly explained, and the facilitator
explained how to work around the issue.

\subsubsection{Tasks}
The four tasks they were asked to complete are listed in 
Figure~\ref{figure:second_tasks}.
As in the first test, the test subjects had access to the definitions for
\texttt{Vect} and \texttt{zip} textual Idris. See Section~\ref{subsec:Idris}
for more on \texttt{Vect} and \texttt{zip}.

\begin{figure}
\centering
\begin{itemize}
	\item \textbf{Task 1}: The user is shown the \texttt{Nat} and \texttt{Fin} data declaration in the program.
	\begin{itemize}
		\item \texttt{T1.1}: Describe the \texttt{Nat} type.
		\item \texttt{T2.2}: Describe the \texttt{Fin} type.
	\end{itemize}
	\item \textbf{Task 2}: Define a data declaration for the vector type.
	\begin{itemize}
		\item \texttt{T2.1}: Specify an identifier for the type (\texttt{Vect}), along with its type (\texttt{Nat -> Type -> Type})
		\item \texttt{T2.2}: Specify the Nil constructor (\texttt{Nil: Vect z a})
		\item \texttt{T2.3}: Specify the Cons constructor (\texttt{(::): a -> Vect k a -> Vect (S k) a})
	\end{itemize}
	\item \textbf{Task 3}: Define the zip function for vector type.
	\begin{itemize}
		\item \texttt{T3.1}: Specify the identifier for the function (\texttt{zip}), along with its type (\texttt{Vect k a -> Vect k b -> Vect k (a, b)})
		\item \texttt{T3.2}: Specify the first case (\texttt{zip Nil Nil = Nil})
		\item \texttt{T3.3}: Specify the second case (\texttt{zip x::xs y::s = (x, y) :: zip xs ys})
	\end{itemize}
	\item \textbf{Task 4}
	\begin{itemize}
		\item \texttt{T4.1}: Move the \texttt{Vect} declaration up below \texttt{Nat}.
	\end{itemize}
\end{itemize}
\caption{Tasks for the second usability test. The text in parentheses are what we considered the correct answer, and was not given to the test subjects.}
\label{figure:second_tasks}
\end{figure}

\subsubsection{Issues}
\label{sec:second_issues}
\todo{Write intro}
\\ \\
\textbf{I1: Too Many many input fields}.
All the users reported that all the grey, unfilled input fields were confusing.
\todo{Insert screenshot}
\\ \\
\textbf{I2: Writing data declaration}
Everyone was able to decipher the data declarations almost immediately, but
all but one had trouble writing the \texttt{Vect} data declaration. Especially
the distinction between the premise area and the conclusion area seemed
problematic
\\ \\
\textbf{I3: Arrows in data confusing}
The use of arrows in the premise area is confusing, as it implies a function.
\\ \\
\textbf{I4: Suggestions for parameterized types}
This issue persists from the original design. The subjects found it irritating
that after choosing a suggestion, they would have to manually change the
parameters using the standard iOS text editing facilities. Multiple subjects
mentioned that it broke their flow.
\\ \\
\textbf{I5: Lack of auto-closing parentheses and quotation marks}
Users found it irritating that they had to manually pair symbols such as
parentheses.
\\ \\
\textbf{I6: Common symbols inaccessible}
Many symbols that are common in programming are hard to find in the iOS virtual
keyboard.
\\ \\
\textbf{I7: Poor flow between input fields}
To move from one input field to the next, the subjects must manually tap each
input field. This broke the subjects flow.
\\ \\
\textbf{I8: Program can look cluttered}
After finishing all the tasks, the subject's program started to look cluttered.
\todo{Insert screenshot}
\\ \\
\textbf{I9: Few definitions visible}
Related to Issue I8, definitions are quite large, so only a few are visible at
any time. This can make it hard for the subject to get an overview of their
work.
\\ \\
\textbf{I10: Unclear how to specify implicits and identifiers for arguments}
Some parts of the syntax have not been determined. Some of the subjects
wondered how one would make an argument implicit.

\subsubsection{Recommendations}
\label{sec:second_recommendations}
\todo{Write intro}
\begin{itemize}
	\item \textbf{Re1} (I1): 
\end{itemize}
