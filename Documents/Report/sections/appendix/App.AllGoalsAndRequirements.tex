%!TEX root = ../../Touch Based Idris.tex 

\paragraph{Goals}
The high-level goals of the project:

\begin{itemize} 
	\item \textbf{G-1}: To design a programming editor that leverages Idris to provide a usable solution for the touch-based iPad device
	(Ta-12).
	\item \textbf{G-2}: The edit-compile-test cycle should be as fast as state of the art iPad programming editors
	(Ta-6, Ta-12).
	\item \textbf{G-3}: Minimize the use of the virtual keyboard\,\cite[pp. 76]{nielsen2013mobile} and ideally only make use of it when inputting identifiers or auto completing terms
	(Ta-1, Ta-11, Ta-12).
	\item \textbf{G-4}: The solution should be eligible for submission to the Apple App Store (Ta-5). 
\end{itemize}

\paragraph{Usability Requirements}

The interface should:
\begin{itemize}     
	\item \textbf{U-1}: Clearly communicate program structure. One way to accomplish this it through the use of visual elements such as shape, color and connecting lines (G-1)
	(Ta-18, Ta-19, Ta-20, Ta-22).
	\item \textbf{U-2}: Make use of touch gestures only when it can improve usability (G-1)
	(Ta-2, Ta-9).
	\item \textbf{U-3}: If using visual elements, avoid the pitfall of some visual languages\,\cite{green1992visual} and ensure that the program structure approximately scales to the same degree as state of the art touch-based editors (G-1)
	(Ta-17, Ta-21).
	\item \textbf{U-4}: Implement accelerators for use by the expert user but invisible for the novice user\,\cite{nielsen1990heuristic}, preferably by use of simple touch gestures (G-1, G-2, G-3)
	(Ta-1, Ta-2, Ta-4).
	\item \textbf{U-5}: Allow the user to recover from syntactical as well as logical errors in a fast manner (G-2)
	(Ta-3, Ta-6, Ta-7, Ta-8).
	\item \textbf{U-6}: Have a clear indication of where the user can add and edit terms without these indications cluttering the interface (see Section~\ref{subsub:Lisping}) (G-1)
	(Ta-10).
\end{itemize}

\paragraph{Functional Requirements}

Specifically the design must:

\begin{itemize}
	\item \textbf{F-1}: Support initial pattern match creation like the Idris Emacs mode\,\cite{Idris:EmacsMode} (G-1, G-2, G-3)
	(Ta-1, Ta-2).
	\item \textbf{F-2}: Support case splitting of pattern variables like the Emacs mode\,\cite{Idris:EmacsMode} (G-1, G-2, G-3) (Ta-1, Ta-2).
	\item \textbf{F-3}: Support a way to let Idris automatically solve a metavariable like the Emacs mode\,\cite{Idris:EmacsMode} (G-1, G-2, G-3) (Ta-1, Ta-2).
	\item \textbf{F-4}: Allow the user to add, remove, and edit basic data types like \texttt{Vect} (G-1).
	\item \textbf{F-5}: Allow the user to add, remove, and edit basic functions like the \texttt{zip} function (G-1).
	\item \textbf{F-6}: Comply with the Apple App Store Review Guidelines\,\cite{AppStoreGuidelines} (G-4)
	(Ta-5).
	\item \textbf{F-7}: Include an overview of the possible language constructs that are applicable in the current context, inspired by TouchDevelop (see Section~\ref{subsub:TouchDevelop}), Lisping (see Section~\ref{subsub:Lisping}), and Eastwest (see Section~\ref{subsub:Eastwest}) (G-1, G-2, G-3)
	(Ta-11, Ta-12, Ta-13, Ta-14).
	\item \textbf{F-8}: Provide better undo support than simply shaking the device (see Section~\ref{subsub:CodeToGo}) (G-2)
	(Ta-3).
	\item \textbf{F-9}: Collapse or minimize top level declarations not currently in focus. This is inspired by TouchDevelop (see Section~\ref{subsub:TouchDevelop}) (G-2, U-3)
	(Ta-15).
	\item \textbf{F-10}: Allow for a faster way of accessing special characters (see Section~\ref{subsub:Textastic}) (G-2)
	(Ta-4).
\end{itemize}