%!TEX root = ../Touch Based Idris.tex
\chapter{Introduction}
\label{sec:Introduction}
In the last few years touch-based interfaces, in the form of smart phones and tablets, have proliferated. These devices often rely on virtual keyboards for text input, which can be cumbersome to use.
This is especially an issue when programming, as the task traditionally is performed with a physical keyboard
that for example provides easy access to special characters.
To our best knowledge, no widespread solution for programming on a touch-based interface exists.

Idris is ``a general purpose functional language with dependent types''\,\cite{brady2013idris}. The expressive nature of the Idris type system makes advanced tool-support possible, such as automatically generating pattern matching on the different possible constructors for a term, or even filling out a term automatically.

The goal of this project is to develop a design for a touch-based programming editor, that leverages Idris to improve the programming experience compared to existing solutions, e.g.\ by minimizing use of the virtual keyboard.
The design will target the Apple iPad, as it is the most prolific tablet computer available today.

To accomplish this, we will first study existing solutions for touch-based programming, together with other programming paradigms that might improve usability for a touch-based design.
We aim to identify and define their usability shortcomings by using existing usability theories.

With the knowledge we have gained from these activities, we will present a set of goals and requirements for our design.
Based on these, we will present a series of design iterations, with changes rooted in usability tests and usability theory.
To further our understanding of the usability implications of our design, a prototype application will be developed.
Appendix\,\ref{chap:RunningTheCode} describes how to run the project.
The last design we present will serve as a base for future development.

\section{Overview}
Chapter\,\ref{chap:Background} describes the technical background of the project along with a brief overview of the usability theory used. Chapter\,\ref{sec:Analysis} analyses a
wide range of existing programming solutions that are relevant to our project and Chapter
\ref{sec:GoalsAndRequirements} leverages this analysis to present a list of
goals and requirements for our solution design. The first design
iteration is presented in Chapter\,\ref{sec:InitialDesign} along with a usability test of mock-ups. Chapter\,\ref{sec:Architecture}
supplies a discussion of our software architecture and forms the basis for the second design iteration, where our iPad prototype is
described together with a second usability test. The third design iteration described in Chapter\,\ref{sec:third_design_iteration} uses all knowledge previously gained to propose our latest design. Finally, Chapter\,\ref{sec:Evaluation}
evaluates our prototype with reference to our goals and requirements, Chapter\,\ref{sec:Reflection} reflects
upon our process, and Chapter\,\ref{sec:Conclusion} concludes the report.
