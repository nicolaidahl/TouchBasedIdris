%!TEX root = ../Touch Based Idris.tex 
\section{Goals and Requirements}
\todo{mention somewhere that we're trying to be better than other tablet based
solutions and not textual editors in general}

\subsection{Goals} The overall goal of this project is to investigate how the
features of the dependently typed language, Idris, can improve the touch-based
programming experience. It is important to underline that the editor we will be
proposing should not be evaluated with reference to state of the art keyboard-
based programming editors. It is not the goal to revolutionize the way we as
humans program in general. The use case for programming on a touch device will most likely be a
mobile one as it is with most use cases for the i-devices \cite[p. 26]{nielsen2013mobile}.

The high-level goals of the project are:

\begin{itemize} 
	\item \textbf{G-1} The edit-compile-test cycle should be as fast as state of the art iPad 
	programming editors.
	\item \textbf{G-2} Minimize the use of the virtual keyboard\cite[pp. 76]{nielsen2013mobile} and
	ideally only make use of it when inputting identifiers or auto completing terms.
	\item \textbf{G-3} The solution must comply with the Apple App Store Review 
	Guidelines\,\cite{AppStoreGuidelines}, which means that no compilation can take place on 
	the client device.
	\item \textbf{G-4} The user interface must take advantage of the platform it is running on. 
	Otherwise it is pointless to have a native app, and one should consider making a web app.
\end{itemize}

\subsection{Usability Requirements} 
In Section \ref{sec:Evaluation} we will evaluate our prototype user interface with reference to Nielsen's 10 Usability
Guidelines\,\cite{nielsen1990heuristic}, and it is an obvious overall usability
requirement that the interface adheres to these. Furthermore, we will have focus on a range of
usability requirements. The interface should:

\begin{itemize}     
	\item \textbf{U-1} Make use of visual elements such as shape, color and connecting lines to 
	communicate program structure.
	\item \textbf{U-2} Take advantage of the nativeness of the app and make use of touch gestures only
	when it can improve usability.
	\item \textbf{U-3} Avoid the pitfall of many visual languages\cite{green1992visual} and ensure 
	that the program structure, however visual, approximately scales to the same degree as state of 
	the art touch-based editors.
	\item \textbf{U-4} Implement accelerators for use by the expert user but invisible for the novice
	user\,\cite{nielsen1990heuristic}, preferably by use of simple touch gestures.
	\item \textbf{U-5} Support a faster way of accessing the special characters that can be used 
	to define operators in Idris (see Section \ref{subsub:Textastic}).
	\item \textbf{U-6} Allow the user to recover from syntactical as well as logical errors in a fast
	manner. This pertains to goal G-1.
	\item \textbf{U-7} Have a clear indication of where the user can add and edit terms without these
	indications cluttering the interface (see Section \ref{subsub:Lisping}).
\end{itemize}


\subsection{Functional Requirements} 
\label{subsec:FunctionalRequirements} 
More specifically the solution must:

\begin{itemize}
	\item \textbf{F-1} Include initial pattern match creating like the Idris Emacs mode \cite{Idris:EmacsMode}
	\item \textbf{F-2} Include case splitting of pattern variables like the Emacs mode \cite{Idris:EmacsMode}
	\item \textbf{F-3} Include a way to let the compiler automatically solve a metavariable like the
	Emacs mode \cite{Idris:EmacsMode}
	\item \textbf{F-4} Allow the user to define basic data types like Vect.
	\item \textbf{F-5} Allow the user to define basic functions like the zip function.
	\item \textbf{F-6} Entail a way to actually run the Idris code.
\end{itemize}

The design of the interface should:

\begin{itemize}     
	\item \textbf{F-7} Include an overview of the possible language constructs that are
	applicable in the current context inspired by TouchDevelop (\ref{subsub:TouchDevelop}), Lisping
	(\ref{subsub:Lisping}), and Eastwest (\ref{subsub:Eastwest}).
	\item \textbf{F-8} Have a logics-inspired standard form of declaring data types as done by Epigram
	(\ref{subsub:Epigram}).
	\item \textbf{F-9} Provide better undo support than simply shaking the device (see Section
	\ref{subsub:CodeToGo}).
	\item \textbf{F-10} Collapse or minimize top level declarations not currently in focus. This is inspired by TouchDevelop
	in Section \ref{subsub:TouchDevelop}.
\end{itemize}


\subsection{Defining Scope} 
The primary goal for the first prototype of IdrisTouch is to test and explore
issues related to defining simple data types and functions with the touch-based interface. 
We have chosen to focus on the definition and manipulation of the canonical Vect data type, which is, as we explained in
section\todo{insert ref}, often used as the basic example of how dependent types
work. Specifically, we will let our usability test subjects define the Vect data type and zip
function (also explained in section\todo{insert ref}). The zip function allows the
testing of automatic initial clause creation, automatic case splitting, and metavariable solving.

Any language features of Idris that are not directly needed to be able to test
the usability of defining the above-mentioned top-level declarations will thus
be omitted. This means we will only not handle language features such as
dependent pairs, explicit types, with notation \todo{insert more?}.



% Main points A pure textual interface is not a good solution (such as
% CodeToGo). You have to optimize for touch Ref: Nielsen - Mobile Usability p.
% 26, p. 41

% The difference between touch and mouse. Impossible to say which one is better,
% so it’s important to design for both


% Low memorability with gestures (p 141) It’s probably not a good idea for us to
% invent too many new gestures, as they are hard to discover and people don’t
% remember them Alternatively gestures can be a power user functionality that
% makes it faster to work with the app. We should put in standard single tap
% ways of doing things in any case (p. 143)



