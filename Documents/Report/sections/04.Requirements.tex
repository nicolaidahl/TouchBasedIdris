%!TEX root = ../Touch Based Idris.tex 
\section{Goals and Requirements}
\todo{mention somewhere that we're trying to be better than other tablet based
solutions and not textual editors in general}

\subsection{Goals} The overall goal of this project is to investigate how the
features of the dependently typed language, Idris, can improve the touch-based
programming experience. It is important to underline that the editor we will be
proposing should not be evaluated with reference to state of the art keyboard-
based programming editors. It is not the goal to revolutionize the way we as
humans program in general.

The high-level goals of the project are:

\begin{itemize}          
	\item G-1 The user interface must take advantage of the platform it is running on. Otherwise it is pointless to have a native app, and one should consider making a web app.
	\item G-2 The edit-compile-test cycle should be as fast as state of the art iPad programming editors.     
	\item G-3 The solution must comply with the Apple App Store Review Guidelines\,\cite{AppStoreGuidelines}, which means that no compilation can take place on the client device.
Goal \end{itemize}

\subsection{Usability Requirements}

\begin{itemize}     
	\item U-1 
	\item U-2 
	\item U-3 
\end{itemize}

How gestures should enhance the programming experience Gestures are hard to
discover (ref), so there should be a way to use the interface without the
advanced gestures


\subsection{Functional Requirements} \label{subsec:FunctionalRequirements} here
goes all specific requirements that describes how a system should do something


\subsection{Defining Scope} The main focus of the usability tests is to explore
issues related to defining a standard data type with the touch-based interface
as well as defining a function to manipulate this data type. We have chosen to
focus on the canonical Vect data type, which is, as we explained in
section\todo{insert ref}, often used as the basic example of how dependent types
work. Furthermore we will be letting our usability test subjects define the zip
function (also explained in section\todo{insert ref}). This function allows the
testing of automatic initial clause creation, automatic case splitting, and rhs
code inference.

Any language features of Idris that are not directly needed to be able to test
the usability of defining the above-mentioned top-level declarations will thus
be omitted. This means we will only not handle language features such as
dependent pairs, explicit types, with notation \todo{insert more?}.



% Main points A pure textual interface is not a good solution (such as
% CodeToGo). You have to optimize for touch Ref: Nielsen - Mobile Usability p.
% 26, p. 41

% The difference between touch and mouse. Impossible to say which one is better,
% so it’s important to design for both

% Our mission is not to convert expert programmers from desktop to mobile.
% Mobile users are often in a hurry and programming is of an immersive nature
% Ref: Nielsen - Mobile Usability p. 26

% The touch keyboard is cumbersome to use pp. 76 In VPls, it can be hard to
% distinguish between language and environment Green, Petre

% Low memorability with gestures (p 141) It’s probably not a good idea for us to
% invent too many new gestures, as they are hard to discover and people don’t
% remember them Alternatively gestures can be a power user functionality that
% makes it faster to work with the app. We should put in standard single tap
% ways of doing things in any case (p. 143)



