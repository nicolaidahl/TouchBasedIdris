%!TEX root = ../Touch Based Idris.tex 
\chapter{Goals and Requirements}
\label{sec:GoalsAndRequirements}

\section{Goals}
The overall goal of this project is to investigate how the features of the dependently typed language, Idris, can improve the touch-based programming experience. 
It is important to underline that the design we will propose is meant to compete with other touch-based editor designs, and not with traditional keyboard based editors.
In other words, we are not trying to revolutionize the way we program, instead we are trying to improve the programming experience when using a touch device.
The use case for programming on a touch device will most likely be a mobile one as it is with most use cases for touch devices\,\cite[p. 26]{nielsen2013mobile}.

The high-level goals of the project are:

\begin{itemize} 
	\item \textbf{G-1} To design a programming editor that leverages Idris to provide a usable solution for the touch-based iPad device.
	\item \textbf{G-2} The edit-compile-test cycle should be as fast as state of the art iPad programming editors.
	\item \textbf{G-3} Minimize the use of the virtual keyboard\,\cite[pp. 76]{nielsen2013mobile} and ideally only make use of it when inputting identifiers or auto completing terms.
	\item \textbf{G-4} The solution should strive to be eligible for submission to the Apple App Store. 
\end{itemize}

\section{Usability Requirements} 
In Section~\ref{sec:Evaluation} we will evaluate our prototype user interface with reference to Nielsen's 10 Usability Guidelines\,\cite{nielsen1990heuristic}, and it is an overall usability requirement that the interface adheres to these. 
Furthermore, we will focus on a range of usability requirements. 
The interface should:

\begin{itemize}     
	\item \textbf{U-1} Clearly communicate program structure. One way to accomplish this it through the use of visual elements such as shape, color and connecting lines. (G-1).
	\item \textbf{U-2} Make use of touch gestures only when it can improve usability. (G-1).
	\item \textbf{U-3} Avoid the pitfall of some visual languages\,\cite{green1992visual} and ensure that the program structure, however visual, approximately scales to the same degree as state of the art touch-based editors. (G-1).
	\item \textbf{U-4} Implement accelerators for use by the expert user but invisible for the novice user\,\cite{nielsen1990heuristic}, preferably by use of simple touch gestures. (G-1, G-2, G-3).
	\item \textbf{U-5} Allow the user to recover from syntactical as well as logical errors in a fast manner. (G-2).
	\item \textbf{U-6} Have a clear indication of where the user can add and edit terms without these indications cluttering the interface (see Section~\ref{subsub:Lisping}). (G-1).
\end{itemize}

\section{Functional Requirements} 
\label{subsec:FunctionalRequirements} 
Besides the usability requirements listed above, the design must also enable several functional requirements. Specifically the design must:

\begin{itemize}
	\item \textbf{F-1} Support initial pattern match creation like the Idris Emacs mode\,\cite{Idris:EmacsMode}. (G-1, G-2, G-3).
	\item \textbf{F-2} Support case splitting of pattern variables like the Emacs mode\,\cite{Idris:EmacsMode}. (G-1, G-2, G-3).
	\item \textbf{F-3} Support a way to let Idris automatically solve a metavariable like the Emacs mode\,\cite{Idris:EmacsMode}. (G-1, G-2, G-3).
	\item \textbf{F-4} Allow the user to add, remove, and edit basic data types like \texttt{Vect}. (G-1).
	\item \textbf{F-5} Allow the user to add, remove, and edit basic functions like the \texttt{zip} function. (G-1).
	\item \textbf{F-6} Comply with the Apple App Store Review Guidelines\,\cite{AppStoreGuidelines}. (G-4).
	\item \textbf{F-7} Include an overview of the possible language constructs that are	applicable in the current context, inspired by TouchDevelop (see Section~\ref{subsub:TouchDevelop}), Lisping (see Section~\ref{subsub:Lisping}), and Eastwest (see Section~\ref{subsub:Eastwest}). (G-1, G-2, G-3).
	\item \textbf{F-8} Provide better undo support than simply shaking the device (see Section~\ref{subsub:CodeToGo}). (G-2).
	\item \textbf{F-9} Collapse or minimize top level declarations not currently in focus. This is inspired by TouchDevelop	(see Section~\ref{subsub:TouchDevelop}). (G-2, U-3).
	\item \textbf{F-10} Allow for a faster way of accessing special characters (see Section~\ref{subsub:Textastic}). (G-2).
\end{itemize}

We will refer to these goals and requirements extensively throughout the rest of this report. Appendix\todo{Insert appendix} includes a list of the these goals and requirements, if the reader wishes to refer to them while reading.

\section{Defining Scope} 
The primary goal for the first prototype of the design is to test and explore
issues related to defining simple data types and functions with the touch-based interface. 
We have chosen to focus on the definition and manipulation of the canonical \texttt{Vect} data type, which is, as we explained in section\todo{insert ref}, often used as the basic example of how dependent types work.
Specifically, we will let our usability test subjects define the \texttt{Vect} data type and \texttt{zip} function (also explained in section\todo{insert ref}).
The \texttt{zip} function allows the testing of automatic initial clause creation, automatic case splitting, and metavariable solving.

Any language features of Idris that are not directly needed to be able to test the usability of defining the above-mentioned top-level declarations will thus be omitted. This means we will only not handle language features such as dependent pairs, explicit types, with notation \todo{insert more?}.
