%!TEX root = ../Touch Based Idris.tex
\section{Requirements}
\label{sec:Requirements}
Based on the initial analysis we will here present various sets of requirements for the new touch-based Idris syntax.

\subsection{Functional Requirements}
\label{subsec:FunctionalRequirements}
here goes all specific requirements that describes how a system should do something

\subsection{Usability Requirements}
More non-functional

How gestures should enhance the programming experience
Gestures are hard to discover (ref), so there should be a way to use the interface without the advanced gestures

\subsection{Defining Scope}
The aim of the project is to explore usability issues related to defining a standard data type with the touch-based interface as well as defining a function to manipulate this data type. We have chosen to focus on the canonical Vect data type, which is, as we explained in section\todo{insert ref}, often used as the basic example of how dependent types work. Furthermore we will be letting our usability test subjects define the zip function (also explained in section\todo{insert ref}). This function allows the testing of automatic initial clause creation, automatic case splitting, and rhs code inference.

Any language features of Idris that are not directly needed to be able to test the usability of defining the above-mentioned top-level declarations will thus be omitted. This means we will only not handle language features such as dependent pairs, explicit types, with notation \todo{insert more?}.